% Options for packages loaded elsewhere
\PassOptionsToPackage{unicode}{hyperref}
\PassOptionsToPackage{hyphens}{url}
%
\documentclass[
]{article}
\usepackage{lmodern}
\usepackage{amssymb,amsmath}
\usepackage{ifxetex,ifluatex}
\ifnum 0\ifxetex 1\fi\ifluatex 1\fi=0 % if pdftex
  \usepackage[T1]{fontenc}
  \usepackage[utf8]{inputenc}
  \usepackage{textcomp} % provide euro and other symbols
\else % if luatex or xetex
  \usepackage{unicode-math}
  \defaultfontfeatures{Scale=MatchLowercase}
  \defaultfontfeatures[\rmfamily]{Ligatures=TeX,Scale=1}
\fi
% Use upquote if available, for straight quotes in verbatim environments
\IfFileExists{upquote.sty}{\usepackage{upquote}}{}
\IfFileExists{microtype.sty}{% use microtype if available
  \usepackage[]{microtype}
  \UseMicrotypeSet[protrusion]{basicmath} % disable protrusion for tt fonts
}{}
\makeatletter
\@ifundefined{KOMAClassName}{% if non-KOMA class
  \IfFileExists{parskip.sty}{%
    \usepackage{parskip}
  }{% else
    \setlength{\parindent}{0pt}
    \setlength{\parskip}{6pt plus 2pt minus 1pt}}
}{% if KOMA class
  \KOMAoptions{parskip=half}}
\makeatother
\usepackage{xcolor}
\IfFileExists{xurl.sty}{\usepackage{xurl}}{} % add URL line breaks if available
\IfFileExists{bookmark.sty}{\usepackage{bookmark}}{\usepackage{hyperref}}
\hypersetup{
  pdftitle={TP Preprocesamiento con Tidyverse},
  pdfauthor={Laura Pérez - Gabriel Raby},
  hidelinks,
  pdfcreator={LaTeX via pandoc}}
\urlstyle{same} % disable monospaced font for URLs
\usepackage[margin=1in]{geometry}
\usepackage{color}
\usepackage{fancyvrb}
\newcommand{\VerbBar}{|}
\newcommand{\VERB}{\Verb[commandchars=\\\{\}]}
\DefineVerbatimEnvironment{Highlighting}{Verbatim}{commandchars=\\\{\}}
% Add ',fontsize=\small' for more characters per line
\usepackage{framed}
\definecolor{shadecolor}{RGB}{248,248,248}
\newenvironment{Shaded}{\begin{snugshade}}{\end{snugshade}}
\newcommand{\AlertTok}[1]{\textcolor[rgb]{0.94,0.16,0.16}{#1}}
\newcommand{\AnnotationTok}[1]{\textcolor[rgb]{0.56,0.35,0.01}{\textbf{\textit{#1}}}}
\newcommand{\AttributeTok}[1]{\textcolor[rgb]{0.77,0.63,0.00}{#1}}
\newcommand{\BaseNTok}[1]{\textcolor[rgb]{0.00,0.00,0.81}{#1}}
\newcommand{\BuiltInTok}[1]{#1}
\newcommand{\CharTok}[1]{\textcolor[rgb]{0.31,0.60,0.02}{#1}}
\newcommand{\CommentTok}[1]{\textcolor[rgb]{0.56,0.35,0.01}{\textit{#1}}}
\newcommand{\CommentVarTok}[1]{\textcolor[rgb]{0.56,0.35,0.01}{\textbf{\textit{#1}}}}
\newcommand{\ConstantTok}[1]{\textcolor[rgb]{0.00,0.00,0.00}{#1}}
\newcommand{\ControlFlowTok}[1]{\textcolor[rgb]{0.13,0.29,0.53}{\textbf{#1}}}
\newcommand{\DataTypeTok}[1]{\textcolor[rgb]{0.13,0.29,0.53}{#1}}
\newcommand{\DecValTok}[1]{\textcolor[rgb]{0.00,0.00,0.81}{#1}}
\newcommand{\DocumentationTok}[1]{\textcolor[rgb]{0.56,0.35,0.01}{\textbf{\textit{#1}}}}
\newcommand{\ErrorTok}[1]{\textcolor[rgb]{0.64,0.00,0.00}{\textbf{#1}}}
\newcommand{\ExtensionTok}[1]{#1}
\newcommand{\FloatTok}[1]{\textcolor[rgb]{0.00,0.00,0.81}{#1}}
\newcommand{\FunctionTok}[1]{\textcolor[rgb]{0.00,0.00,0.00}{#1}}
\newcommand{\ImportTok}[1]{#1}
\newcommand{\InformationTok}[1]{\textcolor[rgb]{0.56,0.35,0.01}{\textbf{\textit{#1}}}}
\newcommand{\KeywordTok}[1]{\textcolor[rgb]{0.13,0.29,0.53}{\textbf{#1}}}
\newcommand{\NormalTok}[1]{#1}
\newcommand{\OperatorTok}[1]{\textcolor[rgb]{0.81,0.36,0.00}{\textbf{#1}}}
\newcommand{\OtherTok}[1]{\textcolor[rgb]{0.56,0.35,0.01}{#1}}
\newcommand{\PreprocessorTok}[1]{\textcolor[rgb]{0.56,0.35,0.01}{\textit{#1}}}
\newcommand{\RegionMarkerTok}[1]{#1}
\newcommand{\SpecialCharTok}[1]{\textcolor[rgb]{0.00,0.00,0.00}{#1}}
\newcommand{\SpecialStringTok}[1]{\textcolor[rgb]{0.31,0.60,0.02}{#1}}
\newcommand{\StringTok}[1]{\textcolor[rgb]{0.31,0.60,0.02}{#1}}
\newcommand{\VariableTok}[1]{\textcolor[rgb]{0.00,0.00,0.00}{#1}}
\newcommand{\VerbatimStringTok}[1]{\textcolor[rgb]{0.31,0.60,0.02}{#1}}
\newcommand{\WarningTok}[1]{\textcolor[rgb]{0.56,0.35,0.01}{\textbf{\textit{#1}}}}
\usepackage{graphicx,grffile}
\makeatletter
\def\maxwidth{\ifdim\Gin@nat@width>\linewidth\linewidth\else\Gin@nat@width\fi}
\def\maxheight{\ifdim\Gin@nat@height>\textheight\textheight\else\Gin@nat@height\fi}
\makeatother
% Scale images if necessary, so that they will not overflow the page
% margins by default, and it is still possible to overwrite the defaults
% using explicit options in \includegraphics[width, height, ...]{}
\setkeys{Gin}{width=\maxwidth,height=\maxheight,keepaspectratio}
% Set default figure placement to htbp
\makeatletter
\def\fps@figure{htbp}
\makeatother
\setlength{\emergencystretch}{3em} % prevent overfull lines
\providecommand{\tightlist}{%
  \setlength{\itemsep}{0pt}\setlength{\parskip}{0pt}}
\setcounter{secnumdepth}{-\maxdimen} % remove section numbering

\title{TP Preprocesamiento con Tidyverse}
\usepackage{etoolbox}
\makeatletter
\providecommand{\subtitle}[1]{% add subtitle to \maketitle
  \apptocmd{\@title}{\par {\large #1 \par}}{}{}
}
\makeatother
\subtitle{Aprendizaje Estadístico}
\author{Laura Pérez - Gabriel Raby}
\date{9/2/2021}

\begin{document}
\maketitle

\hypertarget{ejercicio-1}{%
\section{Ejercicio 1}\label{ejercicio-1}}

En la ciudad de General Roca se llevó a cabo una encuesta con el
objetivo de estudiar el hábito alimenticio y de consumo de alcohol de
los estudiantes de los colegios secundarios. Los datos se encuentran en
el archivo \emph{encuesta.csv}.

\hypertarget{importe-los-datos-y-seleccione-las-columnas-8-al-11-y-47-al-51.-muestre-la-tabla-como-un-tibble-y-luego-como-un-marco-de-datos.-quuxe9-diferencias-observa}{%
\paragraph{1) Importe los datos y seleccione las columnas 8 al 11 y 47
al 51. Muestre la tabla como un tibble y luego como un marco de datos.
¿Qué diferencias
observa?}\label{importe-los-datos-y-seleccione-las-columnas-8-al-11-y-47-al-51.-muestre-la-tabla-como-un-tibble-y-luego-como-un-marco-de-datos.-quuxe9-diferencias-observa}}

El primer segmento de código carga el dataset delimitado por \emph{;}, y
crea un nuevo conjunto de datos \emph{casos\_1\_1} que contiene las
columnas seleccionadas.

\begin{Shaded}
\begin{Highlighting}[]
\CommentTok{# carga de dataset encuesta }
\NormalTok{casos <-}\StringTok{ }\KeywordTok{read_csv2}\NormalTok{(}\StringTok{"data/encuesta.csv"}\NormalTok{)}

\CommentTok{# selección de columnas}
\NormalTok{casos_}\DecValTok{1}\NormalTok{_}\DecValTok{1}\NormalTok{ <-}\StringTok{ }\NormalTok{casos }\OperatorTok
\StringTok{  }\KeywordTok{select}\NormalTok{(}\KeywordTok{c}\NormalTok{(}\DecValTok{8}\OperatorTok{:}\DecValTok{11}\NormalTok{, }\DecValTok{47}\OperatorTok{:}\DecValTok{51}\NormalTok{))}
\end{Highlighting}
\end{Shaded}

Esta sentencia presenta el conjunto \emph{casos\_1\_1} como un
\textbf{tibble}

\begin{Shaded}
\begin{Highlighting}[]
\KeywordTok{as_tibble}\NormalTok{(casos_}\DecValTok{1}\NormalTok{_}\DecValTok{1}\NormalTok{)}
\end{Highlighting}
\end{Shaded}

\hypertarget{a-tibble-559-x-9}{%
\section{A tibble: 559 x 9}\label{a-tibble-559-x-9}}

\begin{verbatim}
 Año Curso Sexo    Edad consumiste_alco~ consumiste_alcoh~ consumiste_alcoh~
\end{verbatim}

\\
1 3 1 Mascu\textasciitilde{} 18 Sí Sí Sí\\
2 4 1 Femen\textasciitilde{} 17 Sí Sí Sí\\
3 1 1 Mascu\textasciitilde{} 13 Sí Sí Sí\\
4 3 1 Mascu\textasciitilde{} 17 Sí Sí Sí\\
5 4 1 Mascu\textasciitilde{} 17 Sí Sí Sí\\
6 3 1 Femen\textasciitilde{} 18 Sí Sí Sí\\
7 4 1 Mascu\textasciitilde{} 17 Sí Sí Sí\\
8 1 1 Mascu\textasciitilde{} 14 Sí Sí Sí\\
9 3 1 Femen\textasciitilde{} 17 No No No\\
10 4 1 Femen\textasciitilde{} 17 Sí Sí Sí\\
\# \ldots{} with 549 more rows, and 2 more variables: \#
consumiste\_alcohol\_esta\_semana , tuviste\_un\_exceso\_de\_alcohol

Y esta sentencia presenta el conjunto \emph{casos\_1\_1} como un
\textbf{data.frame}

\end{document}
